\documentclass{article}
\usepackage[utf8]{inputenc}
\usepackage{graphicx}
\usepackage{titlepic}
\usepackage{hyperref}
\usepackage{xcolor}
\usepackage{amsmath}
\usepackage{textcomp}

\titlepic{\includegraphics[width=\textwidth]{Pictures/robotlab1_final.png}}
\title{Getting started with Kuka Robot Cell}
\author{Espen Teigen }
\date{March 2019}


\begin{document}

\maketitle

    Github: https://github.com/EspenTeigen/Kuka-KR-C4-commissioning

\newpage
\tableofcontents{}
\newpage


\section{Robot System Diagram}
    \begin{figure}[!h]
        \centering
        \includegraphics[scale=0.3]{"Pictures/Chart".png}
        \caption{System overview. It is not how everything is wired, but conceptual how it connects}
    \end{figure}

\newpage
\section{A brief overview of the Robot-cell and it's parts}
    \subsection{Manipulator}
    
     \begin{figure}[!h]
        \centering
        \includegraphics[scale=0.09]{"Pictures/KR_3_AGILUS".jpg}
        \caption{Kuka KR 3 R540 Agilus}
    \end{figure}
    
    
    Ahh... The manipulator, the part that everyone thinks is the robot, but, no. A robot consists of a controller, end-effector and a manipulator, but I am getting ahead of myself.
    \\\\
    The manipulator is the part that delivers the movement of the tool. In it self, it is just a collection of motors and joints, or as the robotics companies like to say "jointed-arm kinematic system". 
    \\\\
    The manipulator has four pneumatic(input/outputs) near the end effector. It has an eight-pin M12 contact on the head, where I have used four of the pins on a force torque sensor.  
    
   
    \newpage
    \textcolor{blue}{\textbf{This manipulator is:}
    \begin{itemize}
        \item 6-axis
        \item Capable of moving 3kg of mass(included tool)
        \item Giving a $\pm 0.02mm$ pose repeatabillity
        \item Weighing 26.5kg without a tool
        \item Not very big. It has a maximum reach of 541mm(without tool)
        \item Silent, less then 68dB
        \item Made for pairing with the KR C4 compact controller
        \item What we would call an assembly robot
        \item Really, really fast(See KR 3 R540 Operating Instructions page 12)
        \item In my opinion, really good looking
    \end{itemize}}
    \\\\
    \textcolor{red}{\textbf{It is not:}
    \begin{itemize}
        \item A pot of petunias falling to the ground
        \item \textbf{A toy. When the manipulator is moving at full speed, it has the power to do some major damage to both humans and equipment. Never try to override the safety system, and keep body parts away from the robot cell. Try to avoid standing in front of the robot cell when the robot system are in operation, if you are not sure about how the robot is moving}
         \item A TARDIS\textsuperscript{TM} traveling in wibbly wobbly          timey wimey space. 
    \end{itemize}}
    \\\\
     \textbf{In the back of the manipulator you will find:}
    \begin{itemize}
        \item Air 1-4
        \item X32, this connector is used for mastering(calibrating) the robot
        \item X76, this connector is used for transferring power and data to the connector on the head of the manipulator
        \item One thick cable(X20), used for transferring power to the motors
        \item One thin cable(X21) used for data communication from the controller to the robot. 
        \item One ground wire, used for equipotential bonding with the controller. 
    \end{itemize}
    
  \newpage
  
        \subsubsection{Force-torque sensor}
        \begin{figure}[!h]
            \centering
            \includegraphics[scale=0.2]{"Pictures/FT300".png}
            \caption{Robotiq FT-300}
        \end{figure}
        
        As the section name implies, this is a force-torque sensor. It gives us 3-axis of force, and 3-axis of torque. 
        
          \begin{figure}[!h]
            \centering
            \includegraphics[scale=0.5]{"Pictures/FT300-diagram".png}
            \caption{Axis of sensitivity}
        \end{figure}
        This gives us the possibility to give the robot sensitivity to it's environment. The FT-300 is placed between the manipulator and end-effector.
        \\\\
        This sensor is sending it's data via modbus to a PLC, that transfers the data to the robot system.
       
\newpage
        
        \subsubsection{Pneumatic's}
        The pneumatics on the robot is made of three part. 
        \begin{figure}[!h]
            \centering
            \includegraphics[scale=0.5]{"Pictures/stanley fatmax OL244".jpg}
            \caption{An Stanley Fatmax OL244 air compressor with pressure regulator}
        \end{figure}
        
        \begin{figure}[!h]
            \centering
            \includegraphics[scale=0.3]{"Pictures/festo".jpg}
            \caption{A 5/2 mono-stable solenoid valve}
        \end{figure}
        
        \begin{figure}[!h]
            \centering
            \includegraphics[scale=0.5]{"Pictures/cylinder".jpg}
            \caption{A SMC CD55N12 pneumatic cylinder}
        \end{figure}
        
        \begin{figure}[!h]
            \centering
            \includegraphics[scale=0.5]{"Pictures/air-flow valve".jpg}
            \caption{And two SMC air-flow valves}
        \end{figure}
        
\newpage
        This setup makes it possible to controll 
        
        
        \subsection{Robot Controller}
        \subsubsection{SmartPad}
        \subsubsection{PC-interface}
        \paragraph{Software}
        \subsubsection{EtherCAT}
        


    \subsection{Control cabinet}
        \subsubsection{Front panel}
        \subsubsection{PLC}
        \paragraph{PC-interface}
        \paragraph{Software}
        \subsubsection{Pneumatic's}
    
    \subsection{Safety}
        \subsubsection{Do's and dont's }
        \subsubsection{Emergency stop's}
        \subsubsection{Light grid}


\end{document}
